
\usepackage{textcomp}
\usepackage[utf8]{inputenc}
\usepackage{amsmath,amsfonts,amsthm,amssymb}

%\usepackage[gray]{xcolor}
\usepackage[svgnames,dvipsnames]{xcolor}
\definecolor{myorange}{rgb}{0,0.375,1}

\usepackage{
	%a4wide,
	ellipsis, mparhack,  %
	booktabs, longtable, %
	ltablex              % tabularx with page breaks
}

\usepackage{makeidx}

\usepackage{fancyvrb}
\usepackage{minted}

\raggedbottom
\fvset{frame=lines,
	framesep=3mm,
	framerule=2pt,
	fontsize=\small,
	rulecolor=\color{myorange},
	formatcom=\color{DarkGreen},
}

\newminted{cpp}{tabsize=4,
	frame=lines,
	framesep=3mm,
	baselinestretch=1.1,
	fontsize=\small
}

% Graphics on titlepage
%\usepackage{eso-pic}

\makeatletter % De-TeX-FAQ 

\usepackage{microtype}

\usepackage{ifpdf} % part of the hyperref bundle
\ifpdf % if pdflatex is used

%set fonts for nicer pdf view
\IfFileExists{lmodern.sty}{\usepackage{lmodern}}
  {\usepackage[scaled=0.92]{helvet}
    \usepackage{mathptmx}
    \usepackage{courier} }
\fi

% the pages of the TOC are numbered roman
% and a pdf-bookmark for the TOC is added
\let\myTOC\tableofcontents
\renewcommand\tableofcontents{
   \pagenumbering{Roman}
   \pdfbookmark[1]{Contents}{}
   \myTOC
   \newpage\thispagestyle{empty}~ %
   \newpage % New page right
   \thispagestyle{empty}
   \clearpage
   \pagenumbering{arabic}}

\usepackage[margin=10pt,format=plain,font=small,labelfont=bf]{caption}

% hyperref for PDF
\usepackage[
    bookmarks,
    colorlinks=true,
 	%bookmarksnumbered,
    bookmarksopen=true,
 	bookmarksopenlevel=1,
 	%
	linkcolor=black,
	citecolor=blue,
	urlcolor=blue,
	filecolor=black,
    anchorcolor=black,
	%
	pdfstartview=FitH,
	pdfpagelayout=OneColumn,
	plainpages=false,
	pdfpagelabels,
    %hypertexnames=false,
    linktocpage,
]{hyperref}

\usepackage{url}
%\usepackage{pdfpages}

\usepackage{ccicons}

%\usepackage[square]{natbib}
%\bibliographystyle{natdin} %plainnat geralpha

\renewcommand\bibname{References} 

% Neue Commands und Umgebungen
\newcommand{\ARP}{\texttt{\textbf\small ARPACK}}
\newcommand{\ARPP}{\texttt{\textbf\small ARPACK++}}
\newcommand{\BLAS}{\texttt{\textbf\small BLAS}}
\newcommand{\LAP}{\texttt{\textbf\small LAPACK}}

% Table styles
\renewcommand{\arraystretch}{1.5}
\newcolumntype{L}{>{\ttfamily}l}

% Enable hyphenation for typewriter font.
\def\textvtt#{\bgroup\ttfamily\hyphenchar\font=45\relax\let\next=}
%\DeclareTextFontCommand{\textvtt}{\ttfamily\hyphenchar\font=45\relax}
\hyphenation{ARrc-Sym-Gen-Eig}
\hyphenation{ARlu-Non-Sym-Std-Eig}